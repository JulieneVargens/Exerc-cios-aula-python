\documentclass[12pt,a4paper]{article}
\usepackage[utf8]{inputenc}
\usepackage{amsmath}
\usepackage{amsfonts}
\usepackage{amssymb}
\usepackage{graphicx}
\usepackage[left=3.0cm,right=2cm,top=3.0cm,bottom=2cm]{geometry}
\author{Juliene Vargens}
\begin{document}
\begin{enumerate}
\item Resolver a integral
$\sum(R)=2\displaystyle{\int\limits_{R}^{\infty}\frac{r \rho(r) dr}{\sqrt{r^2-R^2}}}$, sendo 
$\rho(r)=\frac{\sigma^2_v}{2\pi G r^2}$   .
\end{enumerate}
\textbf{Resolução:}\\
\\
Substituindo $\rho$ na integral, tem-se:
\begin{equation}
\sum(R)=2\displaystyle{\int\limits_{R}^{\infty}\frac{r \sigma^2_v dr}{2 \pi G r^2\sqrt{r^2-R^2}}}
\end{equation}
Simplificando o numerador com o denominador e colocando as constantes para fora da integral,tem-se:
\begin{equation}
\sum(R)=\frac{\sigma^2_v}{\pi G}\displaystyle{\int\limits_{R}^{\infty}\frac{dr}{r\sqrt{r^2-R^2}}}
\end{equation}
Resolvendo a integral indefinida $\displaystyle{\int\frac{dr}{r\sqrt{r^2-R^2}}}$ pelo método de substituição, fazendo:
\begin{equation}
u=r^2 \  \rightarrow \ du=2rdr
\end{equation}

\begin{equation}
\int\frac{du}{2u\sqrt{u-R^2}}
\end{equation}
Fazendo:\\
$v=\sqrt{u-R^2}\\
u=v^2+R^2\rightarrow du=2vdv$
Substituindo na integral (4), tem-se:
\begin{equation}
\int\frac{2vdv}{2(v^2+R^2)\sqrt{v^2+R^2-R^2}}=\int\frac{dv}{\sqrt{v^2+R^2}}
\end{equation}

Fazendo mais uma subtituição:\
$v=Rw\longrightarrow dv=Rdw$, fica-se:
\begin{equation}
\int\frac{Rdw}{R^2w^2+R^2}=\int\frac{Rdw}{R^2(w+1)}=\frac{1}{R}\int\frac{dw}{w+1}
\end{equation}
A integral (6) será:
\begin{equation}
\frac{1}{R}\int\frac{dw}{w+1}=\frac{1}{R}\arctan(w)
\end{equation}
Temos que $w=\frac{v}{R}$, em que $v=\sqrt{u-R^2}$,onde por sua vez $u=r^2$, Logo:
\begin{equation}
w=\frac{\sqrt{r^2-R^2}}{R}
\end{equation}
Substituindo (8) em (7):
\begin{equation}
\frac{1}{R}\arctan(w)=\frac{1}{R}\arctan(\frac{\sqrt{r^2-R^2}}{R})
\end{equation}
Agora, voltando a equação (2), onde se deverá multiplicar a equação (9) pelos termos contantes e substituir os limites de integração que vai de R à $\infty$:
\begin{equation}
\sum(R)=\frac{\sigma^2_v}{\pi GR}\arctan(\frac{\sqrt{r^2-R^2}}{R})]\limits_{R}^{\infty}
\end{equation}
A equação (10) será:
\begin{equation}
\frac{\sigma^2_v}{\pi GR}(\arctan(\frac{\sqrt{\infty-R^2}}{R}-\frac{\sqrt{R^2-R^2}}{R})
\end{equation}
 
\begin{equation}
\frac{\sigma^2_v}{\pi GR}(\arctan(\frac{\sqrt{\infty}-\frac{\sqrt{0})
\end{equation}


\end{document}