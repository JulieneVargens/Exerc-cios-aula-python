\documentclass[11pt]{beamer}
\setbeamercovered{transparent}
\usepackage[utf8]{inputenc}
\usepackage[T1]{fontenc}
\usepackage{lmodern}
\usepackage[portuguese]{babel}
\usepackage{amsmath}
\usepackage{amsfonts}
\usepackage{amssymb}
\usepackage{graphicx,wrapfig,lipsum}
\usetheme{Antibes}
\begin{document}
	\author{Juliene Vargens Ferreira\\Orientador:Armando Bernui}
	
	\title{Lenteamento Gravitacional como ferramenta cosmológica}
	%\logo{\includegraphics[scale=0.05]{on.jpg}}
	%	\documentclass{beamer}
		
	
	%\subtitle{}
	%\logo{\includegraphics[scale=0.05]{on.jpg}}
	\institute{Observatório Nacional}
	%\date{}
	%\subject{}
	\setbeamercovered{transparent}
	%\setbeamertemplate{navigation symbols}{}
	\begin{frame}[plain]
		\maketitle
	\end{frame}
			
\begin{frame}
	\frametitle{Breve histórico}
	\begin{columns}
		\column{0.5\textwidth}
		\begin{itemize}
			\item Antes de Einstein: As massas desviam os fótons, tratadas como massas pontuais.
			\item 1915 a relatividade geral de Einstein preveu para o ângulo de deflexão  um fator de 2 comparado ao valor clássico. Confirmado em 1919 por Eddington e outros durante o eclipse solar.
		\end{itemize}
		\column{0.5\textwidth}
		\begin{figure}
			\includegraphics[scale=0.5]{eclipse.jpg}
			\caption{foto obtida por Eddingston e sua equipe}
			\label{figura1}
			\centering
		\end{figure}
	\end{columns}
\end{frame}

\begin{frame}
	\frametitle{Como ocorre o efeito de lenteamento gravitacional(LG)}
	\begin{columns}
		\column{0.4\textwidth}
		Os raios de luz(fotóns) ao passarem  na vizinhança de um campo gravitacional   mudam sua trajetória: o raio luminoso  sofre um desvio!
		\column{0.6\textwidth}
		\begin{figure}
			\includegraphics[scale=0.45]{ilustracao2.png}
			\caption{ilustração dos raios de luz sendo desviados por um objeto maciço}
			\label{figura2}
			\centering
		\end{figure}
	\end{columns}
\end{frame}

\begin{frame}{Como ocorre o efeito de lenteamento gravitacional(LG)}
	\begin{figure}
		\includegraphics[scale=0.6]{ilustracao1.png}
		\centering
		
	\end{figure}

\end{frame}
\begin{frame}{Arcos no aglomerado de galáxia}
	\begin{figure}
		\includegraphics[scale=0.5]{ilustracao3.png}
		\centering
		\caption{Abell 1689}
	\end{figure}	
\end{frame}

\begin{frame}{Motivação}
oi
\end{frame}
\begin{frame}{Motivação}
 Promissor levantamentos de dados 
\end{frame}
\begin{frame}{Desvio da luz:ângulo de deflexão}
	\uncover<1->{Para um campo gravitacional fraco, a métrica será uma pequena pertubação na métrica de Minkowski,\\
	$$ds^2=(1+\dfrac{2\phi}{c^2})c^{2}dt^2-(1-\dfrac{2\phi}{c^2})(\vec{x})^{2}$$}\\
	\uncover<2->{Uma das formas para obter o ângulo de deflexão é pelo \alert{princípio de Fermat}:O tempo de viagem da luz $T=\int\dfrac{n}{c}dl$ é estacionário,\\
$$\delta(T)=0=\int_{A}^{B}n(\vec{x}(l))dl$$}\\
	\uncover<3->{Geodésica nula(ds=0): $c^{'}=\dfrac{|d\vec{x}|}{dt}\approx c(1+\frac{2\phi}{c^2})$\\
	$$\framebox[4cm]{n=\dfrac{c}{c^{'}}\approx 1-\dfrac{2\phi}{c^2}}$$}
\end{frame}
\begin{frame}{Desvio da luz: ângulo de deflexão}
	Integrando a equação de Euler-Lagrange ao longo do caminho da luz se obtem,\\
	$$ \hat{\vec{\alpha}}=\dfrac{2}{c^2}\displaystyle{\int_{-\infty}^{+\infty}\nabla_{\perp}\phi dl}$$\\
Partícula de massa M: |$\hat{\vec{\alpha}}|=\dfrac{4GM}{c^{2}b}$\\
\begin{figure}
\includegraphics[scale=0.55]{ilustracao6.png}
\centering
\caption{Desvio da luz por uma massa pontual}
\end{figure}

\end{frame}
\begin{frame}{Ângulo de deflexão}
	\uncover<1->{O ângulo de deflexão depende linearmente da da massa M. Isso garante que os ângulos de deflexão de um conjunto de lentes possam ser sobrepostos linearmente.\\
		$$ \hat{\vec{\alpha}}(\vec{\xi})=\sum_{i}^{}\hat{\vec{\alpha}}(\vec{\xi}-\vec{\xi_{i}})= \dfrac{4G}{c^2}\sum_{i}^{}M_{i}\dfrac{\vec{\xi}-\vec{\xi_{i}}}{|\vec{\xi}-\vec{\xi_{i}}|^{2}} $$\\}
	\uncover<2->{Com a transição para densidade contínua,\\
	$$ M_{i}(\vec{\xi_{i}},z)\rightarrow \displaystyle{\int d^2\xi^{'}} \displaystyle{\int dz^{'}\rho(\vec{\xi^{'}},z^')} $$\\}
	\uncover<3->{Onde se tem que $\Sigma(\vec{\xi^{'}})=\displaystyle{\int dz^{'}\rho(\vec{\xi^{'}},z^')}$ é a densidade superficial bidimensional.Assim, $$\hat{\vec{\alpha}}(\vec{\xi})=\dfrac{4G}{c^2}\displaystyle{\int \dfrac{\vec{\xi}-\vec{\xi_{i}}}{|\vec{\xi}-\vec{\xi_{i}}|^{2}} \Sigma(\vec{\xi^{'}}) } $$}
	
 \end{frame}
\begin{frame}{Equação da lente}
	content...
\end{frame}
\begin{frame}
	\frametitle{Equação da lente}
	\begin{columns}
		\column{0.5\textwidth}
		\includegraphics[scale=0.5]{eql.png}
		\caption{foto obtida por Eddingston e sua equipe}
		\label{figura1}
		\centering
		\column{0.5\textwidth}
			$$\vec{\eta}=\frac{D_{s}}{D_{d}}\vec{\xi}-D_{ds}\hat{\vec{\alpha}}$$\\
			Introduzindo coordenadas ângulares,\\
			$$\vec{\eta}=D_{s}\vec{\beta}\ \ \ \ \ \vec{\xi}=D_{d}\vec{\theta}$$\\
			Assim,\\
			$$\vec{\beta}=\vec{\theta}-\vec{\alpha}\ \ \ \ \ \ \ \; \ \vec{\alpha}=\dfrac{D_{ds}}{D_{s}}\hat{\vec{\alpha}}(D_{d)\vec{\theta}$$\\
			sjhajdhkj\\
	\end{columns}
\end{frame}


\begin{frame}
	oi
\end{frame}
\end{document}
